\documentclass{sigchi}

% Use this command to override the default ACM copyright statement
% (e.g. for preprints).  Consult the conference website for the
% camera-ready copyright statement.

%% EXAMPLE BEGIN -- HOW TO OVERRIDE THE DEFAULT COPYRIGHT STRIP -- (July 22, 2013 - Paul Baumann)
% \toappear{Permission to make digital or hard copies of all or part of this work for personal or classroom use is      granted without fee provided that copies are not made or distributed for profit or commercial advantage and that copies bear this notice and the full citation on the first page. Copyrights for components of this work owned by others than ACM must be honored. Abstracting with credit is permitted. To copy otherwise, or republish, to post on servers or to redistribute to lists, requires prior specific permission and/or a fee. Request permissions from permissions@acm.org. \\
% {\emph{CHI'14}}, April 26--May 1, 2014, Toronto, Canada. \\
% Copyright \copyright~2014 ACM ISBN/14/04...\$15.00. \\
% DOI string from ACM form confirmation}
%% EXAMPLE END -- HOW TO OVERRIDE THE DEFAULT COPYRIGHT STRIP -- (July 22, 2013 - Paul Baumann)

% Arabic page numbers for submission.  Remove this line to eliminate
% page numbers for the camera ready copy 
% \pagenumbering{arabic}

% Load basic packages
\usepackage{balance}  % to better equalize the last page
\usepackage{graphics} % for EPS, load graphicx instead 
%\usepackage[T1]{fontenc}
\usepackage{txfonts}
\usepackage{times}    % comment if you want LaTeX's default font
\usepackage[pdftex]{hyperref}
% \usepackage{url}      % llt: nicely formatted URLs
\usepackage{color}
\usepackage{textcomp}
\usepackage{booktabs}
\usepackage{ccicons}

\usepackage{enumerate}

\usepackage{listings}


\usepackage[font=normalsize,labelfont=bf,skip=5pt]{caption}
\usepackage{subcaption}

\usepackage[acronym]{glossaries}

% llt: Define a global style for URLs, rather that the default one
\makeatletter
\def\url@leostyle{%
  \@ifundefined{selectfont}{\def\UrlFont{\sf}}{\def\UrlFont{\small\bf\ttfamily}}}
\makeatother
\urlstyle{leo}

% To make various LaTeX processors do the right thing with page size.
\def\pprw{8.5in}
\def\pprh{11in}
\special{papersize=\pprw,\pprh}
\setlength{\paperwidth}{\pprw}
\setlength{\paperheight}{\pprh}
\setlength{\pdfpagewidth}{\pprw}
\setlength{\pdfpageheight}{\pprh}

% Make sure hyperref comes last of your loaded packages, to give it a
% fighting chance of not being over-written, since its job is to
% redefine many LaTeX commands.
\definecolor{linkColor}{RGB}{6,125,233}
\hypersetup{%
  pdftitle={SIGCHI Conference Proceedings Format},
  pdfauthor={LaTeX},
  pdfkeywords={SIGCHI, proceedings, archival format},
  bookmarksnumbered,
  pdfstartview={FitH},
  colorlinks,
  citecolor=black,
  filecolor=black,
  linkcolor=black,
  urlcolor=linkColor,
  breaklinks=true,
}

\newcommand{\todo}[1]{~\\\\~\textbf{\large{\textcolor{red}{#1}}}\\~\\~\\~\\~}



% define code environment, keeps the listing on the same page, without break
% listing style attributes: https://en.wikibooks.org/wiki/LaTeX/Source_Code_Listings
\lstnewenvironment{code}[1][]%
{\noindent\minipage{\linewidth}\medskip 
	\lstset{
		frame=single,
		numbers=left,
		numbersep=5pt,
		breaklines=true,
		numberstyle=\tiny,
		showstringspaces=false,
		basicstyle=\small\ttfamily,
		captionpos=b,
		#1
	}
}
{\endminipage}

% create a shortcut to typeset table headings
% \newcommand\tabhead[1]{\small\textbf{#1}}


% End of preamble. Here it comes the document.
\begin{document}

\title{From Large Touch Walls to Tablets -- a Responsive Web Framework for Large Multi-Touch Wall Applications}

\numberofauthors{3}
\author{%
  \alignauthor{1st Author Name\\
    \affaddr{Affiliation}\\
    \affaddr{City, Country}\\
    \email{e-mail address}}\\
  \alignauthor{2nd Author Name\\
    \affaddr{Affiliation}\\
    \affaddr{City, Country}\\
    \email{e-mail address}}\\
  \alignauthor{3rd Author Name\\
    \affaddr{Affiliation}\\
    \affaddr{City, Country}\\
    \email{e-mail address}}\\
}

\maketitle

\newacronym{spa}{SPA}{single-page application}
\newacronym{dom}{DOM}{Document Object Model}
\newacronym{rwd}{RWD}{responsive web design}
\newacronym{sp1}{SP1}{sprint planning 1}
\newacronym{sp2}{SP2}{sprint planning 2}
\newacronym{uri}{URI}{uniform resource identifier}


\begin{abstract}
Agile software development, a highly collaborative and communication intensive project management approach, has become a predominant method in modern software development. A core implement for the collaboration and communication are task boards, typically large physical pin boards, which are used for the whole project planning.  Appropriate digital tools and infrastructures to support this intensive collaboration in the team are still missing. 
We developed concepts and prototypes for a web based agile collaboration platform using a large multi-touch wall system, which supports both, team interaction on the wall, as well as direct individual interaction by multiple users using mobile devices like tablets. Combining these two interaction methods raises unique challenges with respect to how to represent the huge amount of information from the large wall system on a small device like a tablet, on the user interface design and the flow of information.
In this paper we present a responsive web design approach that supports application interaction as a team and as an individual using both the large multi-touch wall system as well as tablets.
\end{abstract}

\category{H.5.2.}{Information Interfaces and Presentation (e.g. HCI)}{User Interfaces -- \textit{Interaction styles}}
\category{H.5.3.}{Information Interfaces and Presentation (e.g. HCI)}{Group and Organization Interfaces -- \textit{Computer-supported cooperative work}}
\category{H.5.4.}{Information Interfaces and Presentation (e.g. HCI)}{Hypertext/Hypermedia -- \textit{Architectures}}


\keywords{Collaborative work and interaction; multi-touch; wall; tablets; responsive design; agile;  web components}

\section{Introduction}
% intro to challenges, motivation
In agile software development physical pin boards, so called task boards, are the main tool for  project planning and team collaboration. While a physical card wall offers many advantages with respect to ease of use and flexibility, it also has major disadvantages compared to digital solutions. However, recent studies \cite{udcw:31721, Mateescu:2015, Kropp:2015} show that current desktop-based digital solutions can hardly fulfill the requirements concerning agile collaboration.
Large digital multi-touch wall systems have the potential to be able to replace physical boards and to offer even more possibilities to support agile collaboration: They can foster the collaborative work and generate awareness through digital enrichment of the wall application like pro-active information provisioning, team activation tools, or issue highlighter, for example. A digital solution can also support collaboration of distributed agile teams. However they also must provide the natural interaction and ease of use of physical boards (\cite{udcw:31721} and \cite{Mateescu:2015}). 


We developed a web-based solution for extra large digital multi-touch wall systems, the \textit{aWall} web application, intended to replace the paper-based pin board. 
\begin{figure}
	\centering
	\includegraphics[width=\columnwidth]{figures/awall}
	\caption{The \textit{aWall} application running on a large multi-touch wall composed of 2x2 46" displays.}
	~\label{fig:awall}
\end{figure}

The system consists of two distinct device types: The wall, a 2x2 46"(2.13m x 1.24m) 4k high resolution screen system, as a direct replacement for the pin board (see Figure~\ref{fig:awall}) and tablets for more input-oriented and individual tasks.


While the digital wall seems to be a very well suited platform for team oriented activities like discussions, management of cards, information dissemination, or common design meetings, the tablet is the a much better tool for adding text based information on an individual basis, or adding new cards. It especially allows direct interaction of multiple users simultaneously. 
Thus the two device types complement each other in terms of interaction capabilities and also ease of use. 
However using these two different interaction devices also raises new challenges about how the team via the wall and the individuals with their tablets interact with the whole system. Those challenges include information dissemination, information representation, activity awareness, and information awareness, for example.

% - structure of the paper
The remainder of the paper is structured as follows.
First, the \textit{aWall} web application is presented, followed by related work. Then, the user interface design and the challenges we faced when adapting a web application designed for large screens to smaller tablet-sized screens is presented.
Afterwards, the implementation is described and in the end, we discuss the work and provide an outlook into our future work.


\section{aWall Web Application}

\textit{aWall} is an agile web collaboration plattform. It is designed as a multi-device system for  large multi-touch walls and tablets as shown in Figure~\ref{fig:awall}.

The system is designed to support all different agile meetings including Daily Standup, Sprint Planning 1 and 2 meetings, release planning meeting, as well as retrospectives ( see Figure~\ref{fig:retrospective}) and User Story grooming meetings. 
For some of the meetings a direct interaction with the wall, respectively with the artifacts represented at the wall, seem less appropriate. Other, smaller, input devices like individual developers' tablets or even smartphones might be more appropriate. These devices allow multiple people to directly and independently interact with the artifacts on the wall and enter or modify data during the meeting more comfortable; while everybody immediately seeing the changes made. 

\begin{figure}
	\centering
	\includegraphics[width=\columnwidth]{figures/retrospective}
	\caption{The Retrospective meeting view on the \textit{aWall}.}
	~\label{fig:retrospective}
\end{figure}

The application offers multiple interaction possibilities, ranging from multiple people working on the wall simultaneously to people using the application with tablets and the wall together.
With the server of the system offering real-time change notifications, all connected application instances receive the modifications made by other instances and are able to update the data in-place without reloading the website.

\begin{figure}
	\centering
	\includegraphics[width=\columnwidth]{figures/systemoverview}
	\caption{System overview.}
	~\label{fig:systemoverview}
\end{figure}

The whole system consists of four components, as depicted in Figure~\ref{fig:systemoverview}.
The \textit{aWall} application files are delivered by a web server.
\textit{aWall} is a \gls{spa}, that means the whole application is retrieved in a single page load.
The information about sprints, user stories and tasks is provided by an application server using a REST API and a WebSocket endpoint for real-time change notifications.
The application server in turn retrieves the data from an JIRA~\cite{jira} server. JIRA is a bug, issue and project management software used by many software development teams around the world.

Figure~\ref{fig:awall} shows our setup using 2x2 full-HD monitors forming a 4K display.
Interactions can span monitors and are detected using an infra-red overlay in the frame around the four monitors.

The system organized around the meetings. That means, it offers dedicated views for the different agile meetings, which each serves a different purpose with different goals and different artifacts. For example, the Daily Standup meeting is the daily micro planning of the team, mainly dealing with the individual task cards, while the Sprint planning 1 meeting, is the selection of the User Stiories for the next sprint (iteration).
Each view is composed of one large main widget and several smaller widgets above the main widget.
The main widget is always tailored to the a specific task that coincides with the corresponding meeting.
The smaller widgets offer additional functionality and interaction possibilities and are highly configurable to the users needs.
Figure~\ref{fig:awall} shows the view for the Daily Standup meeting with the extra information views above the main view.


\section{Related Work}
% Scrum taskboards stuff:
The task board for the Daily Standup meeting is one of the core artifacts in agile project management. Thus much  work has been done on this. 
A digital tabletop application is introduced for agile planning meetings by Ghanam et al in \cite{Ghanam:4599452}. 
The tabletop only has a single-touch display. 
Thus only one person can interact with the system at a time and without advanced gestures that involve multiple fingers like pinch-to-zoom.
In \cite{Rubart:2014:CMS:2669485.2669551} Rubart proposes a cooperative multi-touch taskboard for large displays.
The paper discusses how interactive tabletops and surfaces can be applied to the agile taskboard for the daily stand-up and other meetings and evaluates gestures known from smartphones for big tabletops and group settings.
Haas et al \cite{Haas:2014:TAV:2669485.2669538} introduce a task assignment system for tabletop computers that consists of two components: the tabletop application and a separate Android companion application.
The companion application shows changes made on the tabletop in real-time and offers some limited interaction capabilities.

% wall, tablet stuff
A multi-surface communication and collaboration platform using a wall, tabletops and tablets for an emergency operations center is presented by Chokshi in \cite{Chokshi:2014:EMM:2669485.2669520}. 
The system is spatially aware of all the devices and people in the room using multiple Kinects.
This allows the use of gestures like a flick on a tablet towards the wall to share information.
The device types are used for different scenarios:
The tablet is used for personal and private interaction with colleagues.
Whereas the tabletop is a semi-private workspace for collaboration and the wall is a public information radiator, aggregating multiple data sources.
The authors developed native applications for the wall, the tabletop and the tablet.
Thus multiple independent applications were developed with different code bases using in part the same information provided by servers.

% Responsive design:
One of the most significant challenges with the emergence of all the new devices like smartphones, tablets and phablets in the last years, is device and platform fragmentation.
Developing a unique experience for each device would mean to have an application per device type.
This also means multiple times the work if something needs to be changed. 
By using a single code base that serves a multitude of devices, a maintenance nightmare can be avoided.
For the web, HTML5 and CSS3 bring many advancements to create a single code base for different device and screen sizes.
The approach using a single application is called \gls{rwd}.
In \cite{Pandey:2013:RDT:2525194.2525271}, the author presents a case study of a mobile-first design process used to create a hybrid smartphone and tablet application for transaction banking.

% Proximity of user and responsive design:
What responsive design does not take into account is the proximity of the user. 
The authors in \cite{Sukale:2014:PWD:2638728.2638768} developed an example implementation of a responsive web application that adjusts the information displayed according to the user's distance to the display.

%Collaborative web visualization:
In \cite{Badam:2014:PCF:2669485.2669518} Badam presents a web-based application framework for collaborative visualizations across multiple devices. The framework allows to synchronize UI state of new or legacy web applications with minimal overhead.


\section{UI Design and Challenges}
% quick intro with requirements / challenges, focus on SP2 UI as example
Using one application to cover multiple devices with different dimensions presents some extra challenges:

\begin{enumerate}[A.]
	\item \textbf{Device interaction:} How are  the two device types used in the system to work collaboratively.~\label{itm:interaction}
	
	\item \textbf{Responsive design}
	\begin{enumerate}[1.]
			\item \textbf{Retain functionality:} The application should offer the same functionality on the tablet as on the wall.~\label{itm:retainFunc}
					
			\item \textbf{Device-specific viewing experience:} The two device types have significantly different screen resolutions and physical dimensions.~\label{itm:rwdViewingExp}
			
			\item \textbf{Little to no scrolling} The content should be visible without much scrolling.~\label{itm:scrolling}
	\end{enumerate}

	\item \textbf{Information dissemination:} How information is passed between the the devices.~\label{itm:information}
	
	\item \textbf{Single code base and platform independence:} The application should not be restricted to an operating system and should be based on a single code base.~\label{itm:codeBase}
\end{enumerate}

Challenge \ref{itm:codeBase} is more a requirement than a challenge that \textit{aWall} already fulfills because its a web application and there is no server that detects the browser of the client and then renders the website for that device.
\textit{aWall} is a \gls{spa}, that runs completely in the browser and only depends on a server for the data of the sprints, user stories and tasks.
The following subsections are going to explain the aforementioned challenges in detail and how we solved them in \textit{aWall} using the \gls{sp2} meeting as an example.


\subsection{Interaction}
% device interaction (collaborative tasks on the wall, individual text-oriented tasks on the tablet)
% - device interaction: text-input sucks on the wall, thus tablet for input-oriented tasks (interaction)

The wall's strength is to provide overview and team interaction methods like, for example, assign a person to a task or move a task from 'In Progress' to 'Done'. 
Text input, e.g. creating a new task, on the other hand is cumbersome on the wall. 
The \textit{aWall} application on the wall can be operated by multiple people at the same time using the aforementioned interaction methods.
Text-input using the virtual or physical keyboard is possible but limited to one person at a time.
However, the tablet is much more input-friendly, particularly when it supports pen input with handwriting detection.
This is why in our scenario of the \gls{sp2} meeting, the team gathers around the wall with one person handling the wall and everyone else equipped with a tablet.
People with a tablet create the tasks for the currently discussed user story.
The tasks they create show up on the wall immediately after being saved as unassigned tasks (right panel in Figure~\ref{fig:awall-layout}).
Unassigned tasks are specific to our solution and represent the paper-based card written by a team member that has not been discussed by the team and thus has not been assigned to a user story yet.
The unassigned tasks are discussed and then either assigned to user story or discarded by the person at the wall.


\subsection{Responsive Design}

\begin{figure}
	\centering
	\includegraphics[width=\columnwidth]{figures/awall-layout}
	\caption{UI layout of \textit{aWall} with a main widget for \gls{sp2} as seen on the multi-touch wall with a 4K screen resolution.}~\label{fig:awall-layout}
\end{figure}

\subsubsection{Background}
% - responsive web design
\glsreset{rwd} % make sure RWD is printed completely
The \gls{rwd} \cite{Marcotte:2011} is an approach to create one application that provides an optimal viewing experience across a wide variety of different devices with various display-resolutions. 

% cornerstones of rwd
\gls{rwd} builds on the following three cornerstones~\cite{Marcotte:2011}:
\begin{itemize}
	\item Relative units: The style and layout definitions should use relative units like percent instead of fixed units like pixel. 
	A pixel on a 1080p tablet display does not have the same physical size as a pixel on a 4K multi-touch wall system with 2x2 92" displays. 
	
	\item Flexible images: The images are sized in relative units but there may also be multiple versions of the same image. 
	For example, a low resolution image for small-screen devices and a higher resolution image for larger devices.
	
	\item Media Queries: Media queries are the main cornerstone of \gls{rwd}. 
	They allow us to define breakpoints with conditional statements in the stylesheet at which different CSS style definitions take effect.
\end{itemize}

\begin{figure*}
	\centering
	\begin{subfigure}{0.4\columnwidth}
		\includegraphics[width=\textwidth]{figures/widget-titleonly}
		\caption{Title-only view}
		\label{fig:widget-titleonly}
	\end{subfigure}%
	\hfill
	\begin{subfigure}{0.4\columnwidth}
		\includegraphics[width=\textwidth]{figures/widget-default-docked}
		\caption{Default view (docked)}
		\label{fig:widget-default-docked}
	\end{subfigure}
	\hfill
	\begin{subfigure}{0.4\columnwidth}
		\includegraphics[width=\textwidth]{figures/widget-default-undocked}
		\caption{Default view (undocked)}
		\label{fig:widget-default-undocked}
	\end{subfigure}
	\hfill
	\begin{subfigure}{0.4\columnwidth}
		\includegraphics[width=\textwidth]{figures/widget-full}
		\caption{Full view}
		\label{fig:widget-full}
	\end{subfigure}
	\hfill
	\begin{subfigure}{0.4\columnwidth}
		\includegraphics[width=\textwidth]{figures/widget-editmode}
		\caption{Edit mode}
		\label{fig:widget-editmode}
	\end{subfigure}
	
	\caption{The different views for info-view widgets.}
	\label{fig:infoview-widgets}
\end{figure*}

Media Queries have been introduced in CSS3 and became a W3C standard in June 2012 \cite{mediaqueriesW3C}.
They allow the content of a website to adapt to the conditions of the web browser.
The most prominent example of such a condition is the screen resolution.
By defining break-points, using for example the width of the viewport, the statement defines the style of a \gls{dom} element.
% media query example
Let's assume we have a simple element with a title in a $<$h1$>$ element and a text in a $<$p$>$ element.
The media query could define that the $<$p$>$ element is hidden if the height of the viewport is less than 500 pixels.

% mobile-first
An important concept of \gls{rwd} is mobile-first \cite{Wroblewski:2011}. 
It is an approach to give the mobile site a higher priority and design it with the constraints and capabilities of small-screen devices in mind. 
The approach is based on the assumption that adding content to a website designed for small-screen devices is easier than removing content from the large desktop layout to fit the smaller screen.

% difference to normal websites
A big difference between our application and a normal website, the basis for all the examples in books about responsive design and the mobile-first approach ~\cite{Marcotte:2011,Wroblewski:2011}, that needs to be taken into account is how the application is used.
Websites on the Internet we visit on a daily basis are content-oriented and we visit them to consume text and images.
\textit{aWall} on the other hand is interaction-oriented where the user uses the multi-touch screen to interact with artifacts of the application.

% mobile-first and our interaction-oriented application
In \gls{rwd}, the design of a website is optimized to the screen-size.
This also means that the mobile site shows only the most important content, because the available space is very small.
Additional content shown on the desktop website is hidden and only accessible through menus and functionality is sacrificed in order to  keep it organized and legible. 
This works in a content-oriented website, but not an interaction-oriented website.
Especially when the main goal is to provide the same functionality on the tablet as on the large wall (Challenge~\ref{itm:retainFunc}).
The UI should still provide an appropriate viewing experience on the device (Challenge~\ref{itm:rwdViewingExp}).
% we: wall -> tablet "wall-first": how to bring all this information to tablet
The biggest challenge therefore was to bring all the information and functionality from the wall to the much smaller tablet.


\subsubsection{General Layout}

% context
The agile process has multiple collaborative meetings that deal with different aspects of the project.
% Workspace layout (info-view with widgets, workspace menu, main widget)
The UI of \textit{aWall} in Figure~\ref{fig:awall-layout} is built around workspaces configured with independent widgets that can be used by several workspaces.
For each meeting there is a workspace configured with the most appropriate widgets.
Each workspace consists of a main widget that takes most of the available space, numerous independent widgets in a panel at the top of the screen, called info-view, and the workspace-menu with options to configure the workspace in the top-right corner.

% main widget
The main widget occupies the largest part of the UI and is the main working area.
For each meeting of the agile process there is a specialized main widget that shows exactly what is needed and offers tailored interaction methods.
Supplemental interaction with additional artifacts is provided by the widgets in the info-view.
Each main widget is responsible for the responsive behavior of its content to show the appropriate amount of information depending on the available space.

% info view and widgets
The info-view is the horizontal panel at the top with the smaller widgets.
% responsiveness of number of widgets in the layout (min. width for widgets)
The widgets in the info-view can have different widths from standard width to multiple-times the standard-width.
The number of widgets displayed is calculated by the application using a defined minimal width for info-view widgets. 
The broader the viewport, the more widgets can be displayed.
Info-view widgets can be enabled or disabled.
If a currently disabled widget is enabled and the info-view is full, the widget in the last position is replaced with the newly enabled widget.
This ensures that only the calculated number of widgets are displayed which in turn ensures that the widgets always have enough space.
Disabled widgets can be accessed by the workspace-menu, where each widget can be enabled or disabled. 

% widget abilities (dock, resize etc)
The widgets in Figure~\ref{fig:awall-layout} are in a docked position but can be undocked by dragging them out of the info-view.
When a widget is dragged out of the info-view (undocked) the space is closed whereby all widgets to the right move one position to the left.
If a widget is to be docked again, the widget's space is freed up with all widgets to the right moving one position to the right.
With this, the widgets can be freely arranged by the user.
Once undocked, a widget can be moved around freely and resized by either using the mouse or the pinch-to-zoom gesture on a multi-touch display. 

% widgets sizes (title, default, full, edit mode)
The widgets can have multiple views, as depicted in Figure~\ref{fig:infoview-widgets} for different sizes when being resized.
In the docked position in the info-view the widgets show the default view (Figure~\ref{fig:widget-default-docked}) on larger screens. 
% responsiveness of the view (widget height with title only and default size)
On smaller screens like on a tablet, only the title of the widget is shown (Figure~\ref{fig:widget-titleonly}), henceforward called title-only view.
The default size with the same functionality as on larger screens is shown when it is undocked (Figure~\ref{fig:widget-default-undocked}).
Besides the title-only and the default view, a widget can offer an additional full view (Figure~\ref{fig:widget-full}) and an edit mode (Figure~\ref{fig:widget-editmode}).
The full view is intended to show more information on the widget's subject or offer additional interaction functionality.
It is triggered by specifying breakpoints for the width and height.
Once the widget has exceeded both breakpoints while resizing, the full view is shown.
% - buttons when undocked (dock, reset size, enter edit mode)
Once a widget is in the undocked state, a small icon in the top-right corner appears (see Figure~\ref{fig:widget-default-undocked}), that docks the widget immediately when pressed.
When a widget has been resized, even when it has not reached its full view, a 'resize to default' icon is shown, as depicted in Figure~\ref{fig:widget-full} as the second icon from the right.
If a widget has an edit mode, an 'edit' icon is displayed in the top-right corner of the widget.

% workspace menu
% workspace menu: save/reset workspace, list of widgets with toggle to en/disable, list of available meeting workspaces
The workspace-menu is the small button in the top right corner in Figure~\ref{fig:awall-layout}.
It allows to control which widgets are enabled and disabled with toggles, to switch to the other workspaces of different meetings and to save or reset the current workspace configuration.
As mentioned above, the widgets in the info-view can be rearranged.
The new arrangement can be saved by the user or reset to the default.


\subsubsection{SP2 Workspace on a Tablet}

\begin{figure*}
	\centering
	\begin{subfigure}[b]{1\columnwidth}
		\includegraphics[width=\textwidth]{figures/sp2-overview-framed}
		\caption{Overview}
		\label{fig:sp2-overview}
	\end{subfigure}%
	\quad
	\begin{subfigure}[b]{1\columnwidth}
		\includegraphics[width=\textwidth]{figures/sp2-detail-framed}
		\caption{Detailed view}
		\label{fig:sp2-detail}
	\end{subfigure}
	
	\caption{The two views of the split up \gls{sp2} user interface for smaller displays.}\label{fig:sp2-smallscreen-views}
\end{figure*}


% general tablet size
The two device types the system uses, namely the wall and tablets, have significantly different screen resolutions and physical dimensions. An important goal driving our design was the ability to see everything without much scrolling (Challenge~\ref{itm:scrolling}).

% Reduce space of non-essential elements
% title-only view => undock => default view
A lot of vertical space is used up by the info-view widgets when using the default view as on the wall.
They offer auxiliary functionality. But on the tablet it is not necessary that they are completely visible the whole time in their default view.
Thus, we created the aforementioned title-only view (Figure~\ref{fig:widget-titleonly}).
That way, the widgets are still in the same place and visible, but use much less space.
A downside is that they have to be dragged out of the info-view to be usable.
Once the widget is undocked, it enlarges to the default size and it can be used like on the wall.
	
% Viewing experience: focused vs. overview
In our collaborative scenario of the \gls{sp2} meeting, the wall and the tablet serve different needs, thus the viewing experience is different (Challenge~\ref{itm:rwdViewingExp}).
The wall's duty is to give the whole team the overview of the sprint, while the tablet's focus lies with the individual to create new tasks for the currently discussed user story.

% sp2 main widget views
The main widget for the \gls{sp2} meeting, as depicted in Figure~\ref{fig:awall-layout}, shows the \textit{sprint backlog} on the left and a panel on the right for unassigned tasks.
% Retain functionality, responsiveness of sp2 workspace; 2 "views"
On the wall, everything including the \textit{sprint backlog} and the unassigned tasks can be shown simultaneously.
The tablet does not have enough space to show all that information.
But we do not want to sacrifice functionality or remove content to make it fit the smaller screen.
While the complete \textit{sprint backlog} is important on the wall, it is not relevant for creating a task for a user story.
To retain functionality on the tablet (Challenge~\ref{itm:retainFunc}), we split the widget into two views (Figure~\ref{fig:sp2-smallscreen-views}):
The first view (Figure \ref{fig:sp2-overview}) shows all the user stories of the \textit{sprint backlog} in a grid.
When the user clicks, respectively selects a user story, the second view is shown, called detail view (Figure \ref{fig:sp2-detail}).
There, the selected user story is displayed with all its assigned tasks on the left side and the list of unassigned tasks on the right side.
The back-arrow to the right of the user story brings the user back to the overview.


\subsection{Information dissemination}
The information about changes is propagated to all \textit{aWall} application instances in real-time.
So, if someone on a tablet creates a new unassigned task, the new task is sent to all tablets and the wall where it automatically pops up.
The same happens when an user story or task is deleted or changed.


\section{Implementation}

\subsection{Web Components}
%- HTML5 WebComponents to isolate functionality (JS) and style (CSS)
For the Web UI in \textit{aWall} we follow the Web Components  \cite{webcomponents.org} approach,  a set of four standards currently being produced as a W3C specification.
These four specifications offer the following features that can be used together or individually:
\begin{itemize}
	\item Custom Elements~\cite{CustomElements} Define and use new HTML elements. 
	That ranges from simple format-elements to complex elements with functionality like, for example, the HTML5 $<$video$>$ element.
	
	\item HTML Imports~\cite{HTMLImports}: Include and reuse HTML documents.
	Allows to define custom elements in a separate file,  import it and then use those custom elements. 
	Similar to how CSS and Javascript files can be imported.
	
	\item The $<$template$>$ element~\cite{TemplateElement}: Declare inactive \gls{dom}-trees that can be reused to instantiate custom elements.
	
	\item Shadow DOM~\cite{ShadowDOM}: Encapsulates functionality and style in \gls{dom}-trees.
	Normally CSS definitions are global no matter where they are defined.
	By encapsulating the style definitions in a Shadow DOM, they become local and scoped and do not affect any other \gls{dom} elements outside of the Shadow DOM.
\end{itemize}

%- nested, reusable elements instead of MVC
The application is composed of nested custom HTML elements and does not follow the more traditional MVC pattern. 
This approach goes more in the direction of component-based software engineering that emphasizes the separation of concerns.
The custom elements can be defined in separate HTML documents and be reused in multiple web applications by importing the document.
This also means that the elements can be loosely-coupled and independent.


%- Polymer, Interact.js
\subsection{Frameworks}
The \textit{aWall} web application uses Polymer~\cite{polymer}, a library that abstracts Web Components and allows to define custom elements using declarative HTML tags instead of pure Javascript code.
This makes it easy to write new custom elements without the huge amount of boilerplate code that is required when using the native Javascript APIs directly.
For multi-touch gestures like pinch-to-zoom, drag\&drop and resizability of elements, \textit{aWall} uses the interact.js~\cite{interactJs} library.

\subsection{Responsive Widgets}
% how widgets are implemented by the developer developing widgets
All the widgets use the custom element $<$awall-widget$>$ (see Listing~\ref{lst:widget-elements}).
It gives all the widgets the features like drag \& drop and resizability.
The four views of the widgets are also implemented by custom elements, namely $<$awall-widget-title$>$, $<$awall-widget-size-default$>$, $<$awall-widget-size-full$>$ and $<$awall-widget-editable$>$.
Listing~\ref{lst:widget-elements} shows the pattern on how a widget is implemented using the aforementioned elements.
The title for a widget is given as the parameter to the attribute \textit{value} for the element $<$awall-widget-title$>$.
For the other three view elements, the content is more comprehensive than just a string and thus is enclosed in the element's tags as indicated using HTML comment syntax.

\begin{code}[caption={HTML elements for the widget's different views.},label=lst:widget-elements]
<awall-widget>
  <awall-widget-title value="Team" />

  <awall-widget-size-default>
    <!-- Content for default view -->
  </awall-widget-size-default>

  <awall-widget-size-full showHeight="500" showWidth="400">
    <!-- Content for full view -->
  </awall-widget-size-full>

  <awall-widget-editable showHeight="500" showWidth="400">
    <!-- Content for edit mode -->
  </awall-widget-editable>
</awall-widget>
\end{code}

Not all the view elements must be used of course.
A widget does not need to have a full view or an edit mode.
Both are optional and can be configured using the two parameters \textit{showHeight} and \textit{showWidth} to define their breakpoints at which point they activate and the default view is hidden.
The edit mode is not shown automatically, but is represented by an edit button appearing when available as depicted in Figure~\ref{fig:widget-full}.


% how the custom element awall-widget is implemented by the framework
The $<$awall-widget$>$ custom element implements the features that all widgets have and controls when which view is displayed.
Listing~\ref{lst:awall-widget} shows a part of the template of the $<$awall-widget$>$ element definition.
The $<$content$>$ element is used in Shadow DOM and represents an insertion point for \gls{dom}-trees.
That means that everything between the start and end tag of $<$awall-widget-size-default$>$ in Listing~\ref{lst:widget-elements} is inserted in the position of the content tag that selects the $<$awall-widget-size-default$>$ in line 4 of Listing~\ref{lst:awall-widget}.


\begin{code}[caption={Part of the $<$awall-widget$>$'s template using insertion points for the different HTML tags.},label=lst:awall-widget]
<content select="awall-widget-title" />

<div id="extendedContent">
  <content select="awall-widget-size-default" />
  <content select="awall-widget-size-full" />
  <content select="awall-widget-editable" />
</div>
\end{code}


% visibility of the views
All elements in the template in Listing~\ref{lst:awall-widget} except $<$awall-widget-title$>$ are enclosed in an $<$div$>$ element whose visibility is controlled by the media query shown in Listing~\ref{lst:awall-widget-mediaquery}.
As long as the browser's viewport height is smaller than the defined \textit{max-height} breakpoint, only the title of the widget is shown.
When the widget is dragged out of the info-view, the default view becomes visible. This is not controlled by a media query but by Javascript code.
The widget can now be resized using the pinch-to-zoom gesture or the mouse.
During resizing, the width and height of the widget are compared to the \textit{showHeight} and \textit{showWidth} breakpoints defined for the full view and the edit mode.

\begin{code}[caption={Media Query hiding all the views except the title when the screen is small enough.},label=lst:awall-widget-mediaquery]
@media only screen and (max-height: 40rem) {
  #extendedContent {
    display: none;
  }
}
\end{code}

\begin{code}[caption={How the \gls{sp2} main widget decides which view to show.},label=lst:sp2-template]
<h2>Sprintplanning 2</h2>
	
<template if="{{isSmallScreenHeight}}">
  <awall-userstory-cardlist />
</template>
	
<template if="{{!isSmallScreenHeight}}">
  <awall-sprintbacklog />
  <awall-temptasks />
</template>
\end{code}

\subsection{SP2 Widget}
As mentioned before, each widget is responsible for its own responsive behavior.
The main widgets use the $<$awall-widget$>$ but with drag \& drop and resizability disabled.
The main widget for the \gls{sp2} workspace, as shown in Listing~\ref{lst:sp2-template}, has two templates that are controlled by the boolean variable \textit{isSmallScreenHeight}.
The variable is set by the application's Javascript code and is accessible through Polymer's data binding mechanism.
Depending on the value of the variable one of the two inactive templates is activated.
The views are all compartmentalized into their own custom elements and thus the two templates in Listing~\ref{lst:sp2-template} only contain one element (results in Figure~\ref{fig:sp2-overview}) for the small-screen view and two elements for the large-screen view, that results in Figure~\ref{fig:awall-layout}.
The $<$awall-userstory-cardlist$>$ element for small screens handles its behavior to switch to the detailed view (Figure~\ref{fig:sp2-detail}) itself.
The title for the widget at the top of the listing is visible on any screen and thus is not enclosed in a $<$template$>$ element.


\section{Discussion}

\subsection{Hardware}
% 2x2 sucks because of the bezels when interaction requires the user to cross the bezel from one to another screen
Our setup for the multi-touch wall consists of 2x2 monitors.
In between the monitors there is a small border around each monitor as seen in Figure~\ref{fig:awall}.
Interaction across these borders is possible, but they are a haptical obstacle that is not always successfully crossed.

% our 2x2 uses infrared to detect the touch points: when using pinch-to-zoom, the two fingers must have some distance between them or else only one touch-point is detected.
Because of the 2x2 monitor setup, the touches are detected by an additional bezel around the system using infra-red.
The system has some difficulties detecting two distinct touch-points when the distance between two fingers is too narrow.
This is especially noticeable by using the pinch-to-zoom gesture.

Both of these problems can be avoided by using a single high-resolution wall display that uses a capacitive touch-screen instead of infra-red.

\subsection{UI Design}
% orientation of the tablet (portrait, landscape): only landscape supported atm
A nice feature of tablets is that the user can hold it however he wants.
The orientation of the screen changes with how the user holds the device, which is either portrait or landscape mode.
This confronts the designer or developer with another variation of screen resolution that is challenging because the two modes have different aspect ratios.
That often means that the UI for portrait mode is not adequate for the landscape mode.
Our solution is optimized only for landscape mode for all devices.

% virtual keyboard overlays a lot of space and hides the application
On multi-touch devices without a physical keyboard a virtual on-screen keyboard is provided by the system.
The keybaord is not integrated the same way on all systems.
For example on an Microsoft Surface tablet running Widows 8.1, the virtual keyboard overlays everything and the content is not aware of it and thus can not show the relevant content, e.g. a form, in the small space left above the keyboard.
On the other hand, the virtual keyboard in Android is much more integrated into the operating system and thus the content in an application is automatically moved to the part of the screen not hidden by the keyboard.

% hand-writing with pen
A convenient option for text input on a tablet is using a pen instead of the virtual keyboard.
There are not many tablets on the market that support pen input.
One product line that does are the Microsoft Surface Pro tablets.
The support is built into the virtual keyboard application and handwriting detection works pretty well.
The problem with the keyboard hiding content persists with the pen input since it is the same application handling the input.

\subsection{Responsive Design}
% use of RWD (ad/disadvantages) in aWall
In combination with new best practices like \gls{rwd}, enabled by the new standards, the development of new and adaptable web applications becomes more approachable.
Especially media queries allow us to detect and react to changes like the screen resolution that are important for the rendering of content.
They allow us to either hide elements or render them using a different style (e.g. smaller font).
Together with the HTML $<$template$>$ element, this allows us to create inactive \gls{dom}-trees in conditional templates that can be shown or hidden depending on a boolean variable set by a media query.

% the two views of SP2 for small-screen devices
The main widget for the \gls{sp2} workspace presented in this paper uses such conditional templates to present its content depending on the size of the browsers viewport dimensions.
The views including their specific behavior for both display sizes we implemented, namely the large wall and the tablet, have to be developed separately.
There is no silver bullet that generates all the views automatically.
Thanks to the use of custom elements, more complex elements can reuse other custom elements.


\subsection{Implementation}
The framework we developed with \textit{aWall} shows, that current web technologies and emerging new web standards make it possible to build an application that serves different devices with disparate screen resolutions and physical size using a single code base. 

% use of WebComponents (advantages)
Through its design of using loosely-coupled web components, specifically custom elements with Shadow DOM, there are no dependencies between the widgets.
This allows a developer to create new widgets without worrying about whether there are going to be problems with other widgets like  giving an element an ID that has already been used or having its stylesheet definitions (CSS) affect other elements of other widgets.

% WebComponents: rapid development (0.5 -> 1.0), many changes
By working with the newest technologies, there is always the problem with rapidly evolving and changing APIs and libraries.
Since we started to develop \textit{aWall}, the Polymer library evolved and reached version 1.0 with lots of changes.
It is going to take some time to port the current application developed with Polymer 0.5 to the new version. 

% WebComponents only work natively with Chrome, polyfills for other browsers
Another inconvenience of using the newest web technologies is that the number of browsers implementing all those new features is limited.
Currently, \textit{aWall} only runs natively on the Chrome and Chromium web browsers.
Most other browsers are in progress of implementing the specifications for web components.
For all those browsers not supporting the features, there are so called polyfills available.
Polyfills are Javascript libraries, that add the missing APIs to the browser.
One big disadvantage of using polyfills is their lower performance compared to the native implementations.


\section{Conclusion and Future Work}
% conclusion
In this paper we presented how we designed and developed a responsive web application using a single code base that provides the appropriate viewing experience on the large wall as well as the much smaller screen of a tablet without sacrificing functionality.
As an example, we used the \gls{sp2} meeting of the agile process, where multiple people create tasks for a user story on tablets which are then discussed using the wall, to showcase the responsive and collaborative nature of \textit{aWall}.

%results from taskboard UI tests (-> tasks for future work)?
After the design of the large-screen UI, we conducted some live-action interviews with developers to discover if the wall could be something they want to work with and to get input on how to improve it.
%meetings with a distributed team (e.g. show movement when a task is dragged) [similar to PolyChrome]
In the future, we are going to build on the input we received from the interviews and would also like to extend the wall to work with distributed teams, where the movement of a task, for instance, is mirrored on other walls.
% collaborative aspect: see what's being created/ edited
To see in the application a task that is in the process of being created or is being modified at the moment would also be an interesting feature.
It would make the \textit{aWall} application even more collaborative.


\balance{}

% REFERENCES FORMAT
% References must be the same font size as other body text.
\bibliographystyle{SIGCHI-Reference-Format}
\bibliography{references}

\end{document}

%%% Local Variables:
%%% mode: latex
%%% TeX-master: t
%%% End: